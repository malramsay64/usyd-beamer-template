\documentclass[aspectratio=169, 22pt]{beamer}

\usepackage{amsmath}
% slidedefs provides shorthand environments like {points,plain,centre} which also insert a subsection
\usepackage{slidedefs}

\title{The $\tau$ vs $\pi$ argument is really long and interesting}
\subtitle{It is mostly a notational argument}
\date{\today}
\author[Malcolm]{Malcolm Ramsay}

\usetheme{usyd}
% Choose usyd-logobar instead of usyd to have navigation at the top of the slide
%\usetheme{usyd-logobar}

\titlegraphic{USYDTitle}
\titlegraphicbackground{usydred}

\begin{document}

\titleslide

\part{A part of the argument}
\partslide

\section{Argument}

\begin{plain}{Why use $\tau$ when there is $\pi$}
  \begin{block}{Theorem}
    $\tau$ is great when \href{http://blah}{dealing} with circles
  \end{block}

  \begin{enumerate}
    \item<1-> Fourier transforms
      \begin{equation}
        \hat f(\zeta) = \int_{-\infty}^{+\infty} f(x) e^{-2\pi ix\zeta} dx
      \end{equation}
    \item<2-> A simple pendulum
      \begin{equation}
        T \approx 2\pi \sqrt{\frac{L}{g}}
      \end{equation}
  \end{enumerate}
\end{plain}

\begin{points}{Why not?}
  \item A good question.
\end{points}

\section{Another section}

\begin{centre}{Something centred}
	Word
\end{centre}

\end{document}
